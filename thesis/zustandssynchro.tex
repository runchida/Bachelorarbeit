\chapter{Zustandssynchronisierung allgemeiner linearer Systeme beliebiger Ordnung} \label{ch:zustandssynchro}

TEXT

\section{Beschreibung des linearen Gesamtsystems} \label{sec:systembeschreibung lin}
TEXT in Abschnitt

\begin{bsp} \label{bsp:kommtopo}
	Am Beispiel des Kommunikationsgraphen aus \figref{fig:bspkommunikationsgraph} soll dies verdeutlicht werden. Die Anzahl der Agenten ist $N=3$ und die beschreibenden Matrizen sind
%\begin{equation}
%	\vec{A}\left(\mathcal G\right) = \begin{bmatrix}0 & 1 & 0 & 0 \\ 0 & 0 & 1 & 1 \\ 0 & 1 & 0 & 0 \\ 1 & 0 & 1 & 0\end{bmatrix}, \quad \vec{L}\left(\mathcal G\right) = \begin{bmatrix}1 & 0 & 0 & -1 \\ -1 & 2 & -1 & 0 \\ 0 & -1 & 2 & -1 \\ 0 & -1 & 0 & 1\end{bmatrix}.
%\end{equation}
%Aus der gerichteten Kante $(1,2)$ geht hervor, dass der Agent $1$ Information an den Agenten $2$ sendet. Dies bedeutet, dass der Agent $2$ über die Zustandsdifferenz zu Agent $1$ verfügt und diese Information für die Regelung nutzen kann. Weiterhin besitzt der Agent $2$ Kenntnis über die Relativinformation bezüglich Agent $3$, wogegen der Agent $1$ lediglich über die Zustandsdifferenz zu Agent $4$ verfügt.
	\begin{equation}
		\vec{A}\left(\mathcal G\right) = \begin{bmatrix}0 & 1 & 0 \\ 1 & 0 & 1 \\ 1 & 0 & 0\end{bmatrix}, \quad \vec{L}\left(\mathcal G\right) = \begin{bmatrix}2 & -1 & -1 \\ -1 & 1 & 0 \\ 0 & -1 & 1\end{bmatrix}.
	\end{equation}
	Aus der gerichteten Kante $(2,1)$ geht beispielsweise hervor, dass der Agent $2$ Information an den Agenten $1$ sendet. Dies bedeutet, dass der Agent $1$ über die Zustandsdifferenz zu Agent $2$ verfügt und diese Information für die Regelung nutzen kann. Weiterhin besitzt der Agent $1$ Kenntnis über die Relativinformation bezüglich Agent $3$, wogegen der Agent $3$ lediglich über die Zustandsdifferenz zu Agent $2$ verfügt.
\end{bsp}

\begin{figure} [tbh]
	\centering
		\begin{tikzpicture}
		[scale=1.5,>=stealth',thick]
	
		\node[graphnode] (knoten01) at ( 0,1) {1};
		\node[graphnode] (knoten02) at (-0.7,0) {2};
		\node[graphnode] (knoten03) at ( 0.7,0) {3};
	
		\draw [<->]  (knoten01) to (knoten02);
		\draw [->]  (knoten02) to (knoten03);
		\draw [->]  (knoten03) to (knoten01);

\end{tikzpicture}
	\caption{Kommunikationsgraph des betrachteten Beispielsystems}
	\label{fig:bspkommunikationsgraph}
\end{figure}

%\paragraph{Agenten} \label{sec:Agenten}
Die Dynamik der einzelnen Agenten wird durch das Zustandsraummodell \eqref{eq:agenten} beschrieben. Da jeder Agent über die Relativinformation bezüglich seiner Eingangsnachbarn verfügt, wird die Summe der Zustandsdifferenzen zwischen Agent $i$ und seinen Eingangsnachbarn als Ausgang definiert. Die Dynamik lässt sich folglich durch
\begin{subequations} \label{eq:agenten}
	\begin{align}
		\dot{\vec{x}_i} &= \vec{A}\vec{x}_i + \vec{B}\vec{u}_i, \label{eq:agenten 1} \\
		\vec{y}_i &= \sum_{j \, \in \, \mathcal N_\text{in}(i)}^{}{\left(\vec{x}_i - \vec{x}_j\right)} = \sum_{j=1}^{N}{a_{ij}^{'}\left(\vec{x}_i - \vec{x}_j\right)} \label{eq:agenten 2}
	\end{align}
\end{subequations}
ausdrücken. Darin sind $a_{ij}^{'}$ die Elemente der transponierten Adjazenzmatrix $\vec{A}^\top\left(\mathcal G\right)$ des Kommunikationsgraphen. Wie bereits beschrieben sendet ein Agent $i$ Informationen an einen Agenten $j$, wenn $a_{ij}=1$ ist. Für den späteren Entwurf einer Regelung ist es jedoch entscheidend, von welchen Agenten Information empfangen wird. Daher treten in der Ausgangsgleichung die Elemente der transponierten Adjazenzmatrix auf - ein Eintrag $a_{ij}^{'}=1$ bedeutet, das der Agent $i$ Information von Agent $j$ erhält. Es sei darauf hingewiesen, dass die beschriebenen Elemente $a_{ij}^{'}$ nicht mit den Elementen der Systemmatrix $\vec{A}$ der Agenten zu verwechseln sind. Für die Zustände gilt $\vec{x}_i \in \mathbb{R}^n$, die Dimension der Steuervektoren $\vec{u}_i$ ist $p$. Das Gesamtsystem besteht aus $N$ Agenten, wobei die vollständige Steuerbarkeit des Paares $(\vec{A},\vec{B})$, das heißt der einzelnen Agenten, vorausgesetzt wird. 


Zusammenfassend gilt der folgende Satz.
\begin{satz}[Dynamik des Gesamtsystems] \label{thm:gesamtsystem}
%\begin{satz}[Dynamik des Gesamtsystems] \label{thm:gesamtsystem}
	Gegeben sei ein System aus $N$ homogenen linearen Agenten beliebiger Dynamik der Ordnung $n$, die durch
	\begin{align*}
		\dot{\vec{x}}_i &= \vec{A}\vec{x}_i + \vec{B}\vec{u}_i,\\
		\vec{y}_i &= \sum_{j=1}^N{a_{ij}^{'}\left(\vec{x}_i-\vec{x}_j\right)}
	\end{align*}
	beschrieben werden. Die Dynamik des Gesamtsystems wird dann durch die Gleichungen
	\begin{align*}\begin{split} \label{eq:gesamtsystem}
		\dot{\vec{x}} &= \tvec{A}_N \vec{x} + \tvec{B}_N \vec{u}, \\
		\vec{y} &= \ovec{L}_n \vec{x}
	\end{split}\end{align*}
	beschrieben. Darin sind
	\begin{align}
		\vect{x} &= \begin{bmatrix}\vect{x}_1 & \vect{x}_2 & \cdots & \vect{x}_N\end{bmatrix} \in \mathbb{R}^{nN}, \notag\\[10pt]
		\vect{u} &= \begin{bmatrix}\vect{u}_1 & \vect{u}_2 & \cdots & \vect{u}_N\end{bmatrix} \in \mathbb{R}^{pN}, \\
		\vect{y} &= \begin{bmatrix}\vect{y}_1 & \vect{y}_2 & \cdots & \vect{y}_N\end{bmatrix} \in \mathbb{R}^{nN}.
	\end{align}
\end{satz}
